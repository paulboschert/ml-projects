\documentclass[10pt]{extarticle}
\usepackage{amsmath}
\usepackage{amssymb}
\usepackage{fancyhdr}
\usepackage[bottom=1in, top=1in, left=1in, right=1in]{geometry}
\usepackage{graphicx}
\usepackage{here}
\usepackage{subfigure}
\usepackage[T1]{fontenc}
\DeclareGraphicsExtensions{.pdf,.png,.jpg}

\pagestyle{fancy}
\setlength{\headheight}{.5in}
\setlength{\parindent}{0in}
\usepackage{enumitem}
\setlist[enumerate]{itemsep=0mm}

\begin{document}

\rhead{
          Paul Boschert\\
          10/02/2015\\
          CSCI 5622 - Machine Learning: Learnability (HW 4) Discussion \\
      }

\textbf{\textit{1)}} argue about an ordering of hypothesis classes in terms of complexity: hyperplanes through the origin, arbitrary hyperplanes, and axis-aligned rectangles (you can use your experiments as a guide, but simply reporting those numbers is not sufficient; you must make a mathematical argument)
\\
\textit{Solution:} 
\\
\textbf{\textit{2)}} prove that your frequency correctly classifies any training set (up to floating point precision on the computer).
\\
\textit{Solution:} From the additional reading, svmtutorial.pdf: If we choose some number $l$, and find $l$ points that can be shattered, we choose the points to be:
$$x_i = 10^{-i}, i = 1, ..., l$$
Then we specify any labels:
$$y_1, y_2, ..., y_l, y_i \in \{-1, 1\}$$
Then $f(\alpha)$ gives this labeling if we choose $\alpha$ to be:
$$\alpha = \pi(1 + \sum_{i=1}^{l}(\frac{1 - y_i)10^i}{2})$$
\\
Thus the VC dimension of this is infinite.
\\
\textbf{\textit{3)}} Suppose we are classifying real numbers, not integers. The classifier returns positive (1) if the point is greater than the sin function and negative (0) otherwise.
$$h_\omega(x : x \in \mathbb{R} ) \equiv \begin{cases} 1& \mbox{if } \sin(\omega x) \geq 0 \\ 0 & \mbox{otherwise} \end{cases}$$
Give an example of four points that cannot be shattered by this classifier. How does this relate to the VC dimension?
\\
\textit{Solution:} Four equa-distant points cannot be shattered by this classifier.  For example $x_i = 1, 2, 3, 4$, $y_i = 0, 0, 1, 0$.


Despite having a VC dimension that is infinite, there still exist some points that cannot be shattered by $\sin(\omega x)$.  This doesn't say anything about the upper bound because the upper bound says that \textit{any} set of points can't be shattered by the classifier.

\end{document}

